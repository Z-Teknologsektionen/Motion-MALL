%% Definitioner och så


\hypersetup{pdfpagemode=UseNone}


\setlength\cftparskip{-2pt} %Avstånd mellan subsections i ToC
\setlength\cftbeforesecskip{5pt} %Avstånd mellan sections i ToC
\renewcommand{\cftdotsep}{0} %mellanrummet mellan punkterna mellan subsec och sidnummer i ToC
\setlength\cftsubsecindent{1.3cm} %skjuter in subsec lite i ToC
\renewcommand{\cftsecfont}{\bfseries  \fontshape{n}\selectfont \large}
\renewcommand{\cftsecpresnum}{§ }%paragraftecknet i ToC
\renewcommand{\cftsecaftersnum}{.}%paragraftecknet i ToC
\setlength\cftsecnumwidth{1.1cm} %Inte nöjd
\renewcommand{\cftsubsecfont}{\fontshape{it}\selectfont \small}


\usepackage[scaled]{helvet} %font
\renewcommand*{\familydefault}{\sfdefault} %font



\renewcommand{\arraystretch}{1.6} %gör att det inte blir så tight i rutorna
\newcommand{\firstboxwidth}{p{1.33cm}} %Här ändrar man bredden på boxen som innehåller text 
\newcommand{\boxwidth}{p{12.0cm}} %Här ändrar man bredden på boxen som innehåller text


\newcommand{\subsubp}[1]{ & #1 \\ \hline}
\newcommand{\paragraphboxes}[1]{
\begin{center}
    \begin{tabular}{ | \firstboxwidth | \boxwidth |}
    \hline
    #1
    \end{tabular}
\end{center}
}
\newcommand{\paragraphboxeslong}[1]{
\begin{center}
    \tabletail{
    \hline
    }
    \tablelasttail{}
    \begin{supertabular}{ | \firstboxwidth | \boxwidth |}
    \hline
    #1
    \end{supertabular}
\end{center}
}

\titleformat{\section}{\Large\bf}{§ \thesection}{0.3cm}{} %Lägger till paragraftecknet framför kapitlena

%% Hemmagjorda kommandon

\newcommand{\paragrafnummer}{\stepcounter{subsubsection} \arabic{section}.\arabic{subsection}.\arabic{subsubsection}} %Det här kommandot skriver man när man vill skriva ut numret på en subsubparagraf. Observera att den räknar upp. 

\newcommand{\plabel}[1]{\addtocounter{subsubsection}{-1} \refstepcounter{subsubsection}  \label{#1}} %Används för att på ett korrekt sätt skapa en label till en subsubparagraf

\newcommand{\smallitemlist}[1]{\begin{compactitem}#1\end{compactitem} \vspace{-0.4cm}} %passar ändamålet bra
\newcommand{\smallenum}[1]{\begin{compactenum}#1\end{compactenum} \vspace{-0.4cm}} %passar ändamålet bra
\newcommand{\smallitemfill}{\vspace{0.4cm}}

\newcommand{\blankrad}[1][1]{\\[#1\baselineskip]}

